
\chapter{Aufgabe D9}
Für die Umsetzung der Warnfunktion durch eine LED, die einen Druckabfall-/anstieg signalisiert, wurde eine Statemachine verwendet. Diese besteht aus insgesamt fünf Stati, vier die jeweils 0,8 bzw. 1,6s dauern und die LED ein- bzw. ausschalten und dem Off-Status.
Die Variable \glqq status"' steuert die LED. Weitere Variablen sind ausschließlich Hilfsvariablen oder zum Testen.

\begin{figure}[h!]
	\centering
	\includegraphics[width=1\linewidth]{../Graphiken/SOS_state.png}
	\caption{SOS Statemachine}
	\label{fig:SOS_state}
\end{figure}
Für die Statemachine \glqq SOS"' wurden ebenfalls Unittests durchgeführt. Zum einen wird getestet, ob die Statemachine zu jedem Zeitpunkt x im richtigen State ist, dafür wurde eine Debugvaribale \glqq statelocation"' eingeführt.\\
Des Weiteren wird erfolgreich getestet, ob die Statemachine bei einem dauerhaften Failure aktiv bleibt.
\begin{lstlisting}
	@Test
    public void checkAllStatelocationsAndStatesActiveContinues(){}
  \end{lstlisting}
  
  
  Außerdem, ob die Statemachine in den Off-State wechselt, sollte kein Failure nach einem vollständigem Durchlauf mehr vorliegen.
\begin{lstlisting}
    @Test
    public void checkAllStatelocationsActiveContinuesNot(){}
   \end{lstlisting} 
   
   
  Zuletzt wird getestet, ob die Statemachine nicht dauerhaft läuft und solange kein Fehler anliegt im Off-State verweilt.
   \begin{lstlisting}
    @Test
    public void checkAllStatesDeactiv(){}
\end{lstlisting}