
\chapter{Aufgabe D10}
Ein Testen jedes Datenschritts wäre zu aufwendig, deswegen wird zum einen getestet, ob der erste Rechenschritt korrekt durchgeführt wird und ob per Definition verschiedene Zufallszahlen generiert werden.  Außerdem wird geprüft, ob mit gleichem Seed, die gleichen Zufallszahlen generiert werden und so die Ergebnisse für einen Systemtest reproduzierbar sind.
\begin{lstlisting}
package components;

import assertLib.Assert;

static class RandomGeneratorTest {
    
    RandomGenerator randomGen;
    RandomGenerator randomGen2;
    RingBuffer buffer1;
    RingBuffer buffer2;
    real set_vel = 100.0;
		real noiselevel = 5.0;
		integer mySeed = 10;
    
    @Test
    public void generateFirstDataStep() {
	    real erg = randomGen.calc(set_vel, noiselevel, mySeed);
	    Assert.assertNear(erg, 96.14, 0.01);
    }
    
    @Test
    public void generateDifferentNumbers() {
	    real actual;
	    for(i in 0 .. 99){
	        real erg = randomGen.calc(set_vel, noiselevel, mySeed);
	        buffer1.put(erg);
	        actual = buffer1.getIndex(0);
	        for(j in 1 .. 999){
	            	Assert.assertNotEqual(actual, buffer1.getIndex(j));
	            }
	        }
    }
    
    @Test
    public void reproducible(){
	    for(i in 0 .. 999){
	  
	        buffer1.put(randomGen.calc(set_vel, noiselevel, mySeed));     
	        buffer2.put(randomGen2.calc(set_vel, noiselevel, mySeed));
	    }
	    for(j in 0 .. 999){
	        Assert.assertEqual(buffer1.getIndex(j),buffer2.getIndex(j));
	    }
    
    }
}


\end{lstlisting}