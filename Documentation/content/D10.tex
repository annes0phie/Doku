
\chapter{Aufgabe D10}
Hier wird beschrieben, wie der System Driver getestet wurde.\\
Anforderungen:
\begin{itemize}
	\item Zufallsgenerator führt einen Rechenschritt korrekt durch
	\item Der Zufallsgenerator generiert verschiedene Zufallszahlen
	\item Der Zufallsgenerator generiert mit dem gleichen Seed die gleichen Zufallszahlen
\end{itemize}

Korrekte Berechnung:
\begin{lstlisting}
    @Test
    public void generateFirstDataStep() { }
 \end{lstlisting}
 Ein Testen jedes Datenschritts wäre zu aufwendig, deswegen wird zum einen getestet, ob der erste Rechenschritt korrekt durchgeführt wird.\\
 
 Verschiedene Zufallszahlen:
\begin{lstlisting}
    @Test
    public void generateDifferentNumbers() {    }
\end{lstlisting}
Per Definition sollen m unterschiedliche Zufallszahlen generiert werden. Dies wird mit diesem Test erfolgreich getestet.\\
    
Reproduzierbarkeit
\begin{lstlisting}
    @Test
    public void reproducible(){}
\end{lstlisting}
In diesem Test wird erfolgreich getestet, ob die gleichen Zufallszahlen mit dem gleichen Seed mehrfach produziert werden können.