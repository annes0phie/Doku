\chapter{Aufgabe D14}
\subsubsection{Deltazeit}
In der gesamten Umsetzung wurden 10 Sekunden zur Deltagenerierung genutzt. Durch diese 10 Sekunden wurde kleines Rauschen/Noise bei der Erkennung von Reifendruckabfällen unterdrückt. Auch zum Reset wurde deswegen die Einschwingdauer von 10 Sekunden verwendet.
Sollte die Gesamtfunktionalität jedoch tatsächlich im Fahrzeug implementiert werden, sollte eine höhere Deltatime gewählt werden, da Fahrzeiten deutlich höher sind. In der Simulation wurden jedoch nur 10 Sekunden gewählt, um auch ein schnelles Testen zu gewährleisten.

\subsubsection{Änderbarkeit des Codes/Modell}
Gerade in ASCET sind Größen, wie bspw. die Größe des Buffers, nicht variabel und können nur bei der Definition des Buffers festgelegt werden. Es wurde nach Möglichkeiten gesucht, wie das Modell/Code anpassbarer gemacht werden könnte. Es ist jedoch in ASCET nicht möglich gewesen bspw. ein Array mit variabler Länge zu erstellen. 

\subsubsection{Effizienz bei dem Bearbeiten der Aufgaben}
Die Aufgaben sind teilweise unpräzise gestellt, sodass es viel Interpretationsspielraum gibt. Häufig ist es vorgekommen, dass nach erstem Lösen einer Aufgabe eine neue Interpretation entstanden ist, sodass die Aufgabe nochmal überarbeitet werden musste. Beispielhaft sind dafür die Aufgaben D2/D3/D4.\\
Das Programmieren mit ASCET geht im Idealfall schnell. Treten jedoch Fehler auf, sprengen diese meist die eingeplante Zeit, da durch ASCET wenig Hilfestellung zur Lösung der Probleme geleistet wird.\\
Außerdem entstand der Eindruck, dass grafisches Programmieren langsamer und mühsamer ist, als die Implementierung in Skriptsprache. Ein Beispiel ist hierfür der Buffer, der in Skriptsprache umgesetzt wurde, da dies einfacher und übersichtlicher ist.\\

\subsubsection{Berechnung des Mittelwerts/Referenzwerts}
Die Berechnung des Referenz-/Mittelwerts wird durch die Mittlung der vier Deltas ermittelt. Das Problem dabei ist, dass der abweichende Reifen den Mittelwert mit verfälscht, sodass der abweichende Reifen meist mehr als 0,5\% von den anderen Reifen abweichen muss, um 0,5\% vom Mittelwert abzuweichen.\\
Als Alternative wurde versucht den Referenzwert zu einen=m Reifen über die anderen Reifen zu berechnen. Dies erzeugte aber das Problem, dass ein abweichender Reifen meist alle anderen intakten Reifen sehr stark verfälscht, sodass meist immer bei allen Reifen eine Imbalance erkannt wird.\\
Deswegen wurde entschieden, den Referenzwert über alle vier Reifen zu berechnen. Dies empfinden wir zwar nicht als optimal, aber trotzdem als korrekte Lösung da so gewährleistet wird, eine Imbalance bei dem Reifen erkannt wird, der mehr als 0.5\% vom Referenzwert abweicht.\\