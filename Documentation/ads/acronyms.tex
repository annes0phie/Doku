%!TEX root = ../dokumentation.tex

\addchap{\langabkverz}
%nur verwendete Akronyme werden letztlich im Abkürzungsverzeichnis des Dokuments angezeigt
%Verwendung: 
%		\ac{Abk.}   --> fügt die Abkürzung ein, beim ersten Aufruf wird zusätzlich automatisch die ausgeschriebene Version davor eingefügt bzw. in einer Fußnote (hierfür muss in header.tex \usepackage[printonlyused,footnote]{acronym} stehen) dargestellt
%		\acs{Abk.}   -->  fügt die Abkürzung ein
%		\acf{Abk.}   --> fügt die Abkürzung UND die Erklärung ein
%		\acl{Abk.}   --> fügt nur die Erklärung ein
%		\acp{Abk.}  --> gibt Plural aus (angefügtes 's'); das zusätzliche 'p' funktioniert auch bei obigen Befehlen
%	siehe auch: http://golatex.de/wiki/%5Cacronym
%	
\begin{acronym}[YTMMM]
\setlength{\itemsep}{-\parsep}

\acro{API}{Application Programming Interface}
\acro{BDSG}{Bundesdatenschutzgesetz}
\acro{CEP}{Complex Event Processing}
\acro{DEA}{Deterministischer endlicher Automat}
\acrodefplural{DEA}[DEAs]{Deterministische endliche Automaten}
\acro{EDA}{Event Driven Architecture}
\acro{GB}{Gigabyte}
\acro{GFS}{Google File System}
\acro{HDFS}{Hadoop Distributed File System}
\acro{HTTP}{Hypertext Transfer Protocol}
\acro{IDE}{Integrated Development Environment}
\acro{IP}{Internetprotokoll}
\acro{KB}{Kilobyte}
\acro{LTS}{Long Term Support}
\acro{MB}{Megabyte}
\acro{MPI}{Message Passing Interface}
\acro{MRC}{Map Reduce Class}
\acro{NAS}{Network Attached Storage}
\acro{NEA}{Nichtdeterministischer endlicher Automat}
\acrodefplural{NEA}[NEAs]{Nichtdeterministische endliche Automaten}
\acro{NFS}{Network File System}
\acro{OS}{Operating System}
\acro{OSDI}{Operating Systems Design and Implementations}
\acro{PAP}{Programmablaufplan}
\acro{PDF}{Portable Document Format}
\acro{POM}{Project Object Model}
\acro{RFC}{Request for Comments}
\acro{RSA}{Rivest, Shamir und Adleman}
\acro{SAN}{Storage Attached Network}
\acro{SPOF}{Single Point of Failure}
\acro{SSH}{Secure Shell}
\acro{TMG}{Telemediengesetz}
\acro{VM}{Virtuelle Maschine}

\end{acronym}
